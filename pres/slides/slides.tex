\documentclass{beamer}

\usetheme{sintef}
\titlebackground*{images/default}
\usefonttheme[onlymath]{serif}
\setbeamertemplate{footline}[frame number]

\usepackage{amsmath,mathtools,bm}
\usepackage{caption}
\usepackage{subcaption}
%\usepackage{dingbat} % for checkmark
\usepackage{enumerate}

%% \usepackage{xcolor}
%% \usepackage{hyperref}
%% \hypersetup{colorlinks=true,
%%   linkcolor=black,
%%   urlcolor=black,
% }
%\urlstyle{same}

\usepackage{xcolor}
\definecolor{orangeish}{HTML}{FFDF42}
\definecolor{greenish}{HTML}{00A64F}

\hypersetup{colorlinks,linkcolor=,urlcolor=sintefblue}

\usepackage{tikz}
\usetikzlibrary{positioning,calc,arrows.meta,angles}

\def\green #1{\color{greenish}#1}
\def\yellow #1{\color{orangeish}#1}
\def\red #1{\color{red}#1}

%% \def\itemdot{\item[$\circ$]}

\newcommand*{\ditto}{\texttt{\char`\"}}
\newcommand{\norm}[1]{\left\Vert#1\right\Vert}
\newcommand{\ad}[1]{\langle #1 \rangle}

\newcommand{\fracpar}[2]{\frac{\partial #1}{\partial #2}}
%\newcommand{\grad}{\Vec{\nabla}}
\newcommand{\grad}{\nabla}
\newcommand{\dive}{\grad\cdot }
\newcommand{\vect}[1]{\mathbf{#1}}
\newcommand{\mat}[1]{\mathbf{#1}}
\newcommand{\tens}[1]{\boldsymbol{\mathsf{#1}}}
\newcommand{\upw}{\mathsf{upw}}
\renewcommand{\d}{\,\mathrm{d}}
\newcommand{\real}{\mathbb{R}}
\newcommand{\res}{\mathcal{R}}
\newcommand{\acc}{\mathcal{A}}
\newcommand{\flux}{\mathcal{F}}
\newcommand{\src}{\mathcal{Q}}
\newcommand{\neigh}{\mathcal{N}}
\newcommand{\downstr}{\mathcal{D}}
\newcommand{\upstr}{\mathcal{U}}
\newcommand{\ncell}{n_c}
\newcommand{\ndof}{n_{\mathrm{dof}}}
\newcommand{\cell}{\Omega}
\newcommand{\face}{\Gamma}

\newcommand{\Nvec}{\vec{N}}
\newcommand{\jvec}{\vec{j}}
\newcommand{\abs}[1]{\left| #1\right|}


\newcommand{\cli}{c_{\text{Li}^+}}
\newcommand{\Nli}{N_{\text{Li}^+}}
\newcommand{\tli}{t_{\text{Li}^+}}
\newcommand{\zli}{z_{\text{Li}^+}}
\newcommand{\phili}{\phi_{\text{Li}^+}}
\newcommand{\vecNli}{\vec{N}_{\text{Li}^+}}
\newcommand{\dotsli}{\dot{s}_{\text{Li}^+}}
\newcommand{\Dlieff}{D_{\text{Li}^+}^{\text{eff}}}
\newcommand{\kappalieff}{\kappa_{\text{Li}^+}^{\text{eff}}}

\title{\large{Electrochemical-thermal simulations in 3D using BattMo, the Battery Modelling Toolbox}}
\subtitle{\ \\ \ \\}
\date{ LC-BAT-5 Cluster Scientific Symposium on Gen 3bBatteries Barcelona 2023-11-29}
\author{%
  {\bf August Johansson},  Halvor Nilsen, Xavier Raynaud, Francesca Watson\\
  SINTEF Digital\\
  \vspace{1mm}
  Simon Clark, Eibar Flores, Lorena Hendrix\\
  \vspace{-4pt}
  SINTEF Industry}

\begin{document}

{
  \setbeamertemplate{footline}{}
  \maketitle
}


%% \begin{frame}{Physics based battery simulations}

%%   Why?
%%   \begin{itemize}
%%   \item Good understanding of physical processes.
%%   \item Effect of parameters.
%%   \item Simulation based design, optimization and testing.
%%   \end{itemize}

%%   \ \\

%%   But:
%%   \begin{itemize}
%%   \item Difficult for users to get to grips with (need a simulation expert).
%%   \item Difficult to adapt to new setups (geometries, chemistries etc.).
%%   \item Expensive if using commercial software.
%%   \end{itemize}

%% \end{frame}

\begin{frame}{BattMo: An open-source continuum modeling framework for electrochemical devices}

\footnotesize

  \begin{columns}
    \begin{column}{0.6\linewidth}
      \begin{itemize}
      \item Doyle-Fuller-Newman
      \item PxD, x = 2,3,4 % macroscopic and microscopic
      \item Butler-Volmer kinetics % Could probably be changed, but not through JSON files yet
      \item Fully coupled electrochemical -- thermal simulation
      \item Chemistry neutral: Li-ion batteries (LFP, NMC, LNMO, ...), electrolysers for hydrogen production, seawater systems, etc
      \item Geometry neutral: multilayer pouch cells, jelly roll cells, race track cells, coin cells, etc
        %% \item Easy to use, easy to adapt, accurate, fast and free.
      \item Open-source implementations in MATLAB and Julia
      \item At GitHub or \url{www.batterymodel.com}
      \end{itemize}
    \end{column}
    \begin{column}{0.4\linewidth}
      \centering
      %\hspace{-0.5cm}\includegraphics[width=0.7\linewidth]{./images/BattMo_Logo_stacked-crop.png}
      \includegraphics[width=0.6\linewidth]{./images/BattMo_Logo_stacked-crop.png}

      \vspace{1cm}

      \includegraphics[width=0.99\linewidth]{./images/hydralogo.png}

    %\includegraphics[width=0.8\linewidth]{./images/battmo-pillar.png}
    \end{column}
  \end{columns}

  %\centering

  %% \vspace{0.75cm}

  %% \centering
  %% %\href{https://batterymodel.com}{\color{sintefblue}{https://batterymodel.com}}

  %% \url{github.com/BattMoTeam/BattMo}

\end{frame}


\begin{frame}{Continuum model}

  \begin{itemize}
  \item On a continuum level, the state of an electrochemical system is described by
    \begin{itemize}
    \item Concentration of species $i$, $c_i$
    \item Electric potential $\varphi$
    \item Temperature $T$
    \end{itemize}
  \item These can be determined by conservation laws, namely the
    continuity equations for {\bf mass}, {\bf charge} and {\bf energy},
    all of the form
    \begin{align*}
  &    \frac{\partial u}{\partial t} + \nabla \cdot \vec{N} = \dot{s} \\
   &   \text{boundary conditions}
    \end{align*}
  \item Mass flux term $\vec{N}$ describe transport %(diffusion and migration)
  \item Source term $\dot{s}$ describe reactions %(thermodynamic and kinetics)
  \item Have these PDEs for each macroscopic component, plus diffusion in the microscale.
  \end{itemize}

\end{frame}

%% % From slides2
%% \begin{frame}{Example: single-component, non-isothermal model of the electrolyte}

%%   \footnotesize
%%   \begin{itemize}
%%   \item The mass and charge conservation equations are
%%     \begin{align*}
%%       \fracpar{c}{t} + \dive \Nvec_+ &= 0 \\
%%       \dive \jvec &= 0
%%     \end{align*}
%%   \item The mass flux is driven by both {\bf diffusion} and {\bf migration}
%%     \begin{equation*}
%%       \Nvec_+ = - D\grad c + \frac{t_+}{z_+F}\, \jvec
%%     \end{equation*}
%%     where the current density can be approximated as
%%     \begin{equation*}
%%       \jvec = -\kappa\grad\varphi - \kappa\frac{t_+ - 1}{z_+F}\left(\fracpar{\mu}{c}\right)\grad c + \frac{1}{z_+F}\left(\fracpar{\mu}{T}\right)\grad T
%%     \end{equation*}
%%   \item  The equation for the temperature can be derived from non-equilibrium thermodynamics
%%     \begin{equation*}
%%       c_{p}\rho\fracpar{T}{t} = \dive(\lambda\grad T)+\frac{|\,\jvec  \,|^2}{\kappa} + \left(\frac{\partial \mu}{\partial c}\right)\frac{\abs{D\grad{c}}^2}{D}
%%     \end{equation*}
%%   \item The diffusion $D$, electric and thermal conductivities $\kappa$ and $\lambda$ may depend on the state of the system.
%%   \end{itemize}

%% \end{frame}


%% \begin{frame}{Electrochemical continuum models}

%%   \begin{itemize}
%%   \item The state of the electrolyte is described by $(\cli, \phili, T)$
%%   \item The continuity equation for mass is
%%     \begin{align*}
%%       \fracpar{(\cli \varepsilon_l)}{t} + \dive \vecNli &= \dotsli
%%     \end{align*}
%%     where both diffusion and migration drives the transport:
%%     \begin{align*}
%%       \vecNli &= -\Dlieff \grad \cli + \frac{\tli}{\zli F}\, j_l
%%       j_l &=
%%     \end{align*}
%%   \end{itemize}

%%   \begin{align*}
%%     \label{eq:massconselyte}
%%     \fracpar{(\cli \varepsilon_l)}{t} + \dive \vecNli &= 0 \\
%%     \dive \Vec{j} &= 0
%%   \end{align*}
%%   where the mass flux and current density are
%%   \begin{align*}
%%     \vecNli &= - D_e\grad c_e + \frac{\tli}{\zli F} \, \Vec{j} \\
%%     \Vec{j} &= -\kappa \grad \phili - \kappa \frac{1 - \tli}{\zli F} \left(\fracpar{\mu}{\cli}\right)\grad \cli
%%   \end{align*}

%% \end{frame}


\begin{frame}{BattMo = MATLAB Reservoir Simulation Toolbox + electrochemistry expertise}
  \footnotesize

  \begin{columns}
    \begin{column}{0.6\linewidth}
      \centering

      \ \\

      %\hspace{-0.8cm}\includegraphics[width=0.4\linewidth]{./images/mrst-crop.png}
      %\hspace{1.cm}
      %\vspace{0.25cm}
      \begin{itemize}
      \item Available at \url{www.mrst.no} and GitHub
      \item Experience with commercial use of open-source software
      \item Conservative, monotone finite volume schemes
      \item Robust Newton solvers
      \item General 3D grids
      \item Adjoint-based optimization
      \item Automatic differentiation (AD):
      \begin{align*}
        \ad{f, df} + \ad{g, dg} &= \ad{f+g, df+dg} \\
        \ad{f, df} \cdot \ad{g, dg} &= \ad{f\,g,\, f\, dg + g\, df} \\
        \exp({\ad{f,df}}) &= \ad{\exp({f}), \, df \exp({f})}
      \end{align*}
      \end{itemize}
    \end{column}
    %\vrule{}
    \begin{column}{0.4\linewidth}
      \centering

      %\vspace{0.5cm}
      \includegraphics[width=0.7\linewidth]{./images/mrst-crop.png}

       \vspace{0.5cm}

       \includegraphics[width=0.6\linewidth]{./images/battmo-pillar.png}
      \includegraphics[width=0.5\linewidth]{./images/battmotext.png}



    \end{column}
  \end{columns}

\end{frame}


\begin{frame}{Implementations in MATLAB and Julia}

  \begin{columns}[T]
    \begin{column}{0.55\textwidth}
      {\bf BattMo} (MATLAB/GNU Octave)
      \begin{itemize}
      \item Mature version
      \item Includes temperature effects, electrolyzers, etc
      \item Large overhead for smaller models
      \item Need a MATLAB licence (although Octave typically works with performance loss)
      \end{itemize}
    \end{column}
    \begin{column}{0.55\textwidth}
      %% \begin{center}
      %%   \includegraphics[width=0.5\linewidth]{./images/julialogo.png}
      %% \end{center}
      {\bf BattMo.jl} (Julia)
      \begin{itemize}
      \item Julia is a free and open-source programming language. JIT-compiled like MATLAB \& Python, but with speed of C.
      \item Fewer features (eg.\ no thermal yet)
      \item Faster simulation times, particularly for small (P2D) models (10x)
      \item Currently uses model setup from MATLAB/Octave.
      \item \texttt{Using Pkg; Pkg.add(\ditto BattMo\ditto)}
      %\item \url{www.github.com/BattMoTeam/BattMo.jl}
      \end{itemize}
    \end{column}
  \end{columns}

\end{frame}

%% \begin{frame}{BattMo implementation}

%%   \begin{columns}
%%     \begin{column}{0.5\textwidth}
%%       \begin{itemize}
%%       \item AD - automatic differentiation
%%       \item Advanced unstructured gridding and visualisation
%%       \item Gradient based optimisation inbuilt
%%       \item MATLAB: easy syntax, nice debugger, great for prototyping
%%       \item Open-source:  easy to see what's going on in the code
%%       \item MATLAB + open-source: easy to adapt to new chemistries
%%       \end{itemize}
%%     \end{column}
%%     \begin{column}{0.5\textwidth}
%%       AD fig; gridding; optimization \& calibration is the same thing
%%     \end{column}
%%   \end{columns}

%% \end{frame}

\begin{frame}{Using BattMo}

  \begin{columns}
    \begin{column}{0.4\textwidth}
      \begin{itemize}
      \item JSON based input format -- code mimics JSON structure
        % Code is modular and chemistry independent to minimize changes not covered by JSON schemas.
      \item Library of JSON files for various electrodes, separators, materials etc provided  % because parameterization takes a lot of work
      \item Actively working with the Battery Interface Ontology for describing the data
      %% \item HDF5 output
      %% \item Data described by the Battery Interface Ontology (BattINFO)\footnote{Eibar Flores A01-0112 - Semantic Technologies to Model Battery Data and Knowledge, Wed.\ 17.40}
      \item Documentation and example
        tutorials to get started with
      \item Possible to deploy in the cloud
      \end{itemize}
    \end{column}
    \begin{column}{0.7\textwidth}
      \centering
      \includegraphics[width=1.1\linewidth]{./images/gui.png}
      \footnotesize
      Web-based GUI in progress (deliverable Nov 2023!)
    \end{column}
  \end{columns}

%% \vspace{1cm}

%%   \centering
%%   %\href{https://batterymodel.com}{\color{sintefblue}{https://batterymodel.com}}
%%   \url{https://batterymodel.com}\hspace{1cm}
%%   \url{https://github.com/BattMoTeam/BattMo}


\end{frame}



%% \begin{frame}{JSON format mimics implementation}

%%   \vspace{2mm}

%%   \centering
%%   \includegraphics[width=0.8\linewidth]{./images/classes.png}

%%   \footnotesize
%%   \begin{itemize}
%%   \item Code is modular and chemistry independent to minimize changes not covered by JSON schemas.
%%   \item A graph-based visualization tool can show how data is passed.
%%   \end{itemize}

%% \end{frame}

\begin{frame}{Comparison with other frameworks}

  % pybamm faster than Matlab battmo, but slower than julia battmmo

  \begin{columns}
    \begin{column}{0.5\textwidth}
      \begin{itemize}
      \item Good agreement with PyBaMM\footnotemark \
        using the Chen 2020 model\footnotemark \ (isothermal, NMC 811 / SiGr).
      \item For some applications P2D is adequate.
      \item For some applications, richer spatial fidelity is of importance.
      \end{itemize}
    \end{column}
    \begin{column}{0.5\textwidth}
      \centering
      \includegraphics[width=0.99\linewidth]{./images/pybamm.png}
    \end{column}
  \end{columns}

\footnotetext[1]{Sulzer, et al.(2021). Journal of Open Research Software,9 (1), 14}
\footnotetext[2]{Chen et al 2020 J. Electrochem. Soc. 167 080534}

\end{frame}

\begin{frame}{Example: Multilayer LFP pouch cell (5 layers)}

  \centering
  \includegraphics[width=0.47\linewidth]{./images/lfpgrid1.png}\hspace{0.5cm}
  \includegraphics[width=0.47\linewidth]{./images/lfpgrid2.png}

\end{frame}

\begin{frame}{How does the temperature behave when increasing the number of layers?}

  \begin{figure}
    \centering
    \hspace{-10mm}
    \begin{subfigure}[b]{0.35\linewidth}
      \centering
      \includegraphics[width=\linewidth]{./images/lfptemp1.png}
      \caption{5 layers}
    \end{subfigure}
    \begin{subfigure}[b]{0.35\linewidth}
      \centering
      \includegraphics[width=\linewidth]{./images/lfptemp2.png}
      \caption{10 layers}
    \end{subfigure}
    \begin{subfigure}[b]{0.35\linewidth}
      \centering
      \includegraphics[width=\linewidth]{./images/lfptemp3.png}
      \caption{100 layers}
    \end{subfigure}
  \end{figure}

\end{frame}

%% \begin{frame}{4680 Jelly roll cell}

%%   \centering
%%   %\includegraphics[width=0.2\linewidth]{./images/fullcellgeo.png}
%%   \includegraphics[width=0.2\linewidth]{./images/tabs.png}\hspace{0.5cm}
%%   \includegraphics[width=0.35\linewidth]{./images/fullgridZoomTop.png}\hspace{0.65cm}
%%   %\includegraphics[width=0.3\linewidth]{./images/temperature.png}
%%   \includegraphics[width=0.25\linewidth]{./images/temperature-top-crop.png}
%%   %\includegraphics[width=0.25\linewidth]{./images/temperature-side-crop.png}

%%   % Easy to test effect of various tab configurations

%%   \begin{itemize}
%%   \item 100k cells and 468k degrees of freedom
%%   \item Good preconditioners is critical for performance when solving large problems
%%   \end{itemize}

%% \end{frame}


\begin{frame}{Calibration and optimization: Same mathematics!}

  \begin{itemize}
  \item  Example of calibration (data assimilation, fitting, parameterization) problem:
    Minimize the difference between the experimental and
    simulated discharge curve
    \begin{align*}
      J(p) = \int_0^T |E(p) - E_\text{exp}|^2 \ dt
    \end{align*}
  \item Example of optimization problem: Maximize the specific energy
    \begin{align*}
      J(p) = \frac{1}{m(p)} \int_0^T E(p)I  \  dt
    \end{align*}
  \item In both cases we (basically) want to find
  \begin{align*}
    p^* = \arg \min_p J(p)
  \end{align*}
  \item This is an optimization problem.
  \end{itemize}

\end{frame}

\begin{frame}{How does one solve the optimization problem most effectively?}

  \begin{itemize}
  \item {\bf Gradient-based optimization} are for problems when
    evaluating the system is costly (large, time-dependent,
    nonlinear).
  \item How the sensitivities  $\nabla_p\, J(p)$ look like require inside information of the
    solver (AD).
  \item If we have a black-box solver, we could use a finite difference approximation, but this scales proportionally to the number of parameters.
    {\bf Unfeasible} for large problems and/or many parameters.
  \item The adjoint approach yield a computational cost {\bf independent}
    of the number of parameters.
  \item Adjoint approach provide {\bf general systematic framework} for optimization
    and calibration.
  \end{itemize}

\end{frame}

%% \begin{frame}{Calibration and optimization: Same mathematics!}
%%   \footnotesize

%%   \vspace{3mm}

%%   Choose an objective function, eg.\ if $E$ is the
%% cell voltage, $I$ a discharge current and $m$ is the mass of the cell,
%%   \begin{itemize}
%%     \item {\bf Calibration}: Minimize the difference between the experimental and
%%       simulated discharge curve
%%       \begin{align*}
%%         \int_0^T |E(p) - E_\text{exp}|^2 \ dt
%%       \end{align*}
%%     \item {\bf Optimization}: Maximize the specific energy
%%       \begin{align*}
%%         \frac{1}{m(p)} \int_0^T E(p)I  \  dt
%%       \end{align*}
%%   \end{itemize}

%% \  \\

%% Given an initial guess, loop until the objective does not change:
%%   \begin{enumerate}
%%   \item Compute of the objective function (requires a full simulation) for the current $p$.
%%   \item Compute of the gradient of the objective function with respect to $p$ (using AD).
%%   \item Use the gradient to {\bf find the optimum direction} to search in for new $p$.
%%   \end{enumerate}

%% \end{frame}


\begin{frame}{Electrode calibration in practice: exploit the possibility to split the parameter space}
\footnotesize
  1. Calibrate equilibrium parameters using {\bf low C} discharge curves.

  \begin{itemize}
  \item Neglect diffusion and transport processes as they are instantaneous.
  \item Assume uniform electrode concentration, only dependent on time.
  \item Fit thermodynamic parameters eg.\ stoichiometry at fully charged state, active material volume fraction, maximum Li conc.\ by solving an gradient-based optimization problem.
  \item Cheap, since no PxD solve is needed.
  \end{itemize}

\ \\

  2. Calibrate rate parameters using {\bf high C} discharge curves.

  \begin{itemize}
  \item Use parameters from the low-rate calibration (1).
  \item Use the optimization framework to fit parameters (eg.\ volumetric surface area, electrical conductivity) % grids in progress
  \item This is a PDE-constrained optimization problem.
  \item Adjoint-based optimization critical in the case of many parameters.
  \end{itemize}

\end{frame}

\begin{frame}{Calibration of an LNMO cell}

  % \centering
  %% \includegraphics[width=0.47\linewidth]{./images/h1.png}\hspace{0.5cm}
  %% \includegraphics[width=0.47\linewidth]{./images/h2.png}

  \begin{figure}
    \centering
    \captionsetup[subfigure]{justification=centering}
    %\hspace{-10mm}
    \begin{subfigure}[]{0.45\linewidth}
      \centering
      \includegraphics[width=\linewidth]{./images/h1.png}
      \caption{Discharge at 0.05 C. Calibration under equilibrium condition approximation.}
    \end{subfigure}
    \begin{subfigure}[]{0.45\linewidth}
      \centering
      \includegraphics[width=\linewidth]{./images/h2.png}
      \caption{Discharge at 2 C. Calibration with adjoint-based gradient optimization. }
    \end{subfigure}
  \end{figure}

\end{frame}



\begin{frame}{Conclusions}

\footnotesize

\begin{columns}%[T]
  \begin{column}{0.6\linewidth}
    \begin{itemize}
    \item {\bf BattMo} is a framework for primarily PxD type of electrochemical --thermal systems
    \item MATLAB and Julia
    \item Flexible chemistry and geometry
    \item Tutorial and library of JSON data files provided
    \item Efficient parameterization and calibration
    \item Freely available at GitHub and \url{www.batterymodel.com}
    \item Experienced in commercial use of open-source software
    \item Interested in collaborations and contributors!
    \end{itemize}
  \end{column}
  \begin{column}{0.4\linewidth}
    \centering
    \includegraphics[width=0.6\linewidth]{./images/BattMo_Logo_stacked-crop.png}
    \scriptsize
  \end{column}
\end{columns}

\vspace{0.5cm}

{
\tiny
%\includegraphics[width=0.25\linewidth]{images/EN_Co-fundedbytheEU_RGB_POS.png}
\includegraphics[width=0.25\linewidth]{images/EN_FundedbytheEU_RGB_POS.png}

This project has received funding from the European Union’s Horizon
2020 innovation programme under number:
\begin{itemize}
\item 957189 Battery interface genome and materials acceleration platform (BIG-MAP)
\item 875527 Hybrid power-energy electrodes for next generation lithium-ion batteries (HYDRA)
\item 101104013 Battery management by multi-domain digital twins (BATMAX)
\end{itemize}
}


\end{frame}

\backmatter


\end{document}
